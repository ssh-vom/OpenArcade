\documentclass[a4]{article}
\usepackage{listings}
\usepackage{LJMU-API}
\usepackage{url}
\usepackage{booktabs}
\usepackage{listings}
\usepackage{fancyvrb}
\usepackage{fancyhdr}   
\usepackage{graphicx}
\usepackage[normalem]{ulem}
\usepackage{varwidth}
\usepackage[x11names]{xcolor}
\usepackage{amsmath, stackengine}
\usepackage{cancel}
\usepackage{xcolor}
\usepackage{hyperref}
\usepackage{biblatex}
\usepackage{tabularx}
\usepackage{array}
\usepackage{enumitem} 
\usepackage{pdfpages}
\usepackage{parskip}
\usepackage{tikzpagenodes} 
\usepackage{eso-pic}       
\addbibresource{sample.bib}
\usepackage{listings}

\newcolumntype{Y}{>{\centering\arraybackslash}X} % horizontal centering
\newcolumntype{Z}{>{\centering\arraybackslash}m{0.45\linewidth}} % optional fixed width


\definecolor{McMasterMaroon}{RGB}{122,0,60}
\definecolor{codegreen}{rgb}{0,0.6,0}
\definecolor{codegray}{rgb}{0.5,0.5,0.5}
\definecolor{codepurple}{rgb}{0.58,0,0.82}
\definecolor{backcolour}{rgb}{0.97, 0.97, 0.97}

\AddToShipoutPictureFG{%
  \begin{tikzpicture}[remember picture,overlay]
    % Bar
    \fill[McMasterMaroon] (current page.south west) rectangle ([yshift=2cm]current page.south east);
    % Centered text inside the bar
    \node[anchor=south east, text=white, font=\ttfamily\bfseries\Large, xshift=-.75cm, yshift=0.65cm] 
      at (current page.south east) {O p e n A r c a d e};
  \end{tikzpicture}%
}

\lstdefinestyle{mystyle}{
    backgroundcolor=\color{backcolour},   
    commentstyle=\color{codegreen},
    keywordstyle=\color{blue},
    numberstyle=\tiny\color{codegray},
    stringstyle=\color{codepurple},
    basicstyle=\ttfamily\footnotesize,
    breakatwhitespace=false,         
    breaklines=true,                 
    captionpos=b,                    
    keepspaces=true,                 
    numbers=left,                    
    numbersep=5pt,                  
    showspaces=false,                
    showstringspaces=false,
    showtabs=false,                  
    tabsize=2
}

\lstset{style=mystyle}

\hypersetup{
    colorlinks=true,
    linkcolor=gray!100,
    filecolor=blue,      
    urlcolor=blue,
    pdftitle={Overleaf Example},
    pdfpagemode=FullScreen,
    }
\input{LJMU-API-header}

\macid{OPENARCADE}
\coursename{MECHTRON 4TB6}
\assignment{Project Goals and Development Process}
\date{\today}

\begin{document}
\begin{titlepage}

	\hspace*{0mm}\textbf{\Large OpenArcade, Group 3}\\
	\hspace*{0mm} \large Anish Paramsothy, paramsa\\
	\hspace*{0mm} \large Chris Palermo, palerc1\\
	\hspace*{0mm} \large Mitchel Cox, coxm12\\
	\hspace*{0mm} \large Shivom Sharma, shars119\\
	\hspace*{0mm} \large Jacqueline Leung, leungw18

	\hspace*{\textwidth}
	\hspace{-4cm}
	\vspace{-2cm}
	\raisebox{1cm}[0cm][0cm]{\includegraphics[width=4cm]{m24-col_png.png}}

	\vspace{7cm}


	\begin{center}
		\Huge \textbf{Project Goals and Development Process} \\[1em]
		\Large MECHTRON 4TB6, McMaster University
	\end{center}
	\vspace{7cm}

	\hspace*{0mm} \textbf{\large Date Submitted:} September 27th, 2025

	\hspace*{0mm} \textbf{\large Due Date:} September 28th, 2025
\end{titlepage}
\tableofcontents

\clearpage
\section{Glossary}
Below is a list of definitions to ensure there is no confusion when reading the document:
\begin{itemize}
	\item \textcolor{McMasterMaroon}{Module}: A unit that contains electronics (boards and wiring), that will communicate inputs to another device.
	\item \textcolor{McMasterMaroon}{Parent Module}: Describes the central hub that children modules will communicate to. This parent module will connect to the computer/console to communicate to the game.
	\item \textcolor{McMasterMaroon}{Child Module}: Describes a unit that may contain joysticks, buttons, d-pads, etc. The child modules can be attached to each other to form a controller, or remain separate. These will communicate the inputs to the parent module and then to the game.
	\item \textcolor{McMasterMaroon}{Parent Board}: Specified board used in the parent module.
	\item \textcolor{McMasterMaroon}{Child Board}: Specified board used in the children modules.
\end{itemize}
\section{Project Goals}
\subsection{Project Description}
Design and development of an arcade/box style controller series of children modules (with a children module containing: buttons, joysticks, d-pad, etc), that can be mechanically
connected to each other to allow for gamers to develop their own combination and style of controller. Along with the idea that gamers can combine modules together,
we want to give them the option to play using the modules separately. These modules will be connected to a parent module which will communicate the inputs of the
modules to the game.

\subsection{Rationale}
Currently, there are several game controllers that follow the ideology of “1 size fits all”, preventing gamers that may be injured, disabled, or lacking fine motor
skills to play the games they want. This could be the case due to a variety of reasons:
\begin{itemize}
	\item The controller itself is too small and is required to be held.
	\item Several controllers require a grip that becomes uncomfortable to the player after hours of gaming.
	\item Buttons and joysticks are close together and require the user to place their hands in positions that can potentially be uncomfortable.
	\item There is little room for customization, and a lot of controllers follow the one ideology of "one size fits all".
\end{itemize}

\subsection{Goals}
We have several goals that we want to outline for the controller. We want to accomplish these to some degree of effectiveness.
\subsubsection{Level 1 Goals}
These goals are the most important for the functionality of the controller.
\begin{itemize}
	\item \textcolor{McMasterMaroon}{Modularized}: The controller should include multiple children modules, and for these to work together to play games.
	\item \textcolor{McMasterMaroon}{Customizable}: The controller should be able to be attached to each other (if A and B are children modules, then they are to be attached in either A-B or B-A formats), or be able to be utilized as separate and detached children modules.
	      The joysticks should also be swappable and include additional customization options.
	\item \textcolor{McMasterMaroon}{Input Delay}: Producing the lowest amount of input delay as possible, to prevent any scenarios where games become unplayable because the inputs are not responsive enough.
	\item \textcolor{McMasterMaroon}{Comfort}: The controller should be attempting to be an ergonomic substitute to other controllers, so providing better options for buttons and joysticks and their placement should be a heavy consideration.
	      Along with this, the location of where the controller will be (either in lap or on table) will also provide an improved gaming experience for the user.
	\item \textcolor{McMasterMaroon}{Functional}: The children modules should be able to fit correctly together without any mechanical issues. Along with this, the controller should function correctly in terms of gameplay, where specific inputs (button presses or joystick movement) relate to the correct outputs.
	\item \textcolor{McMasterMaroon}{Robust}: The controller should meet the basic needs of staying together and not falling apart when subject to stress. This could include the force created when pressing a button, or potentially the user resting their arm on the device while playing.
	\item \textcolor{McMasterMaroon}{Intuitive}: The controller should not be confusing to use, the children modules should attach in a simple manner to prevent confusion.
	\item \textcolor{McMasterMaroon}{Connectivity}: The controller should be able to connect to different types of devices, with the main priority being computer.
\end{itemize}
\subsubsection{Level 2 Goals}
These goals are added optional goals that we want to strive towards to optimize the product.
\begin{itemize}
	\item \textcolor{McMasterMaroon}{Aesthetically pleasing}: The controller should look nice in order to draw in consumers.
\end{itemize}
\section{Development Process}
\subsection{Meetings}
Meetings are planned to be held weekly during 3 core timeslots:
\begin{itemize}
	\item Tuesdays from \textcolor{McMasterMaroon}{11:00AM to 12:00PM}
	\item Thursdays from \textcolor{McMasterMaroon}{4:00PM to 6:00PM}
	\item Fridays from \textcolor{McMasterMaroon}{12:00PM to 1:00PM}
\end{itemize}
These timeslots are meant for planning individual tasks and goals for the following meetings. The time slots are to be considered an open time to meet, but not a requirement. Group discussion outside of these meetings will be conducted to determine the next appropriate meeting date.

The group will also strive to meet at different times as well, such as the weekend, or whenever everyone is available.

Group members are meant to come to meetings with some individual work done, and all work completed is to be documented in meetings notes on Google Drive.
\subsection{Overall Process Workflow}
Below is a list of steps that outline the workflow in which the OpenArcade controller will be designed and fabricated.
\begin{enumerate}
	\item \textcolor{McMasterMaroon}{Discussion of Hazard Analysis}: Meet in a few group meetings to identify hazards related to all aspects of design, including hazards during development, and hazards during use. This will all be noted in a hazard analysis document. To complete this task, all possible hazards should be identified.
	\item \textcolor{McMasterMaroon}{Development of System Requirements}: Outlining the hardware/software/mechanical required to complete the various milestones of the project. This document will be modified during the entire timeline of the project, as the group will define what is needed for the \textit{proof of concept}, \textit{rev 0} and \textit{rev 1} milestones. The document will include a list of software and hardware used, along with the material that will be used for the housing. This will aid in drafting a bill of materials (BOM).
	\item \textcolor{McMasterMaroon}{Concept Design 1}: Drafting an initial concept design to show as the proof of concept. This should include showing the functionality of buttons on a simple scale. The goal is to develop a written plan of what the proof of concept should be, and finalizing what hardware will be required for that.
	\item \textcolor{McMasterMaroon}{Proof of Concept}: The proof of concept will be used to outline basic functionality of the design. This will include:
	      \begin{itemize}
		      \item Ability to communicate between parent board and computer.
		      \item Functionality of children boards (attached to buttons, joysticks, etc) to a breadboard with LEDs
		      \item BOM including hardware choices
	      \end{itemize}
	      The proof of concept will be successful if the button and joystick inputs correspond to the correct outputs on the breadboard. Along with this, another successful test will be if the parent board correctly communicates to the computer.
	\item \textcolor{McMasterMaroon}{Concept Design 2}: Begin planning the finalized design. The group will work in determining how the children modules (with buttons, joysticks, etc) will communicate with the parent module, so that direct inputs can result in outputs seen on the computer. A rough mechanical design will be made with cardboard or other simple materials so that the placement of the buttons and joysticks can be determined. Essentially a primarily housing for the controller. The children modules will not be attachable yet.
	\item \textcolor{McMasterMaroon}{Rev 0}: Showcase the buttons and joysticks communicating to the game itself. The demo will show how the buttons can be configured to specific outputs, and that they communicate the correct outputs for the corresponding inputs. Children modules will communicate to the parent module, which will be communicated to the game. Begin development of button configuration app.
	\item \textcolor{McMasterMaroon}{Concept Design 3}: Discussion and planning of the finalized design. This will include the mechanical housing for all modules, along with how they will be attached to each other to form the full controller. The housing should contain the electrical and hardware components, while also preventing the movement/shifting of the internal pieces. Children modules should be attachable to each other. Emitted heat from the controller should also be considered when developing the housing.
	\item \textcolor{McMasterMaroon}{Rev 1}: The finalized design of the controller. The demo will include two children modules that can be attached to each other, and can successfully communicate with the game through the parent module with the correct outputs. The controller will be compared to other controllers to showcase why it is an effective substitute for the controllers in the market. Finalize development of button configuration app.
\end{enumerate}


\clearpage
\subsection{Coding Standards}

\subsubsection{General Practices}

We will aim to write modular code with unit and integration tests,
following a functional coding approach.
We will avoid writing redundant code and avoid regressions where possible.
Our workflow will be Agile workflow, delineating tasks flexibly and pivoting when necessary.
All code will exist on a single repository, allowing us to maintain simplicity for writing, testing, building
and deploying any new code. We will aim to document our code with comments where necessary,
using readable function names while producing larger pieces of documentation for interactions between
multiple pieces of software or hardware interactions.

\subsubsection{Version Control}

We will be using GitHub as our main version control platform. Changes will be made as "features", we will constrain changes to be single additions where possible, keeping different components decoupled.
Each feature will be a branch and changes will be reviewed in pull requests. When the relevant reviewers for the code have approved,
we will squash and rebase, then merge the changes into the main branch as the source of truth.



\subsubsection{Testing Plan}

Our testing methodology will go hand in hand with our feature creation. Each change should be accompanied by unit tests for the smallest unit of code that is testable in our case a function.
We will try to avoid redundant tests, and each of these tests will be run in a GitHub Actions Workflow. The workflow will automate testing before changes can be merged, this will occur by spinning up
a Docker container that runs the tests in a virtual environment to check that there are no regressions from added changes based on our expected behaviour. If our expectations were wrong we will aim to correct the tests. We will be using the following libraries for testing:

\begin{table}[h!]
	\centering
	\begin{tabular}{|c|c|}
		\hline
		\textcolor{McMasterMaroon}{\textbf{Language}} &
		\textcolor{McMasterMaroon}{\textbf{Testing/Build Tools}}                   \\
		\hline
		Python                                        & pytest, strawberry, codeQL \\
		\hline
		TypeScript/JavaScript                         & Jest, Bun, npm             \\
		\hline
		C/C++                                         & gtest, Make/Bazel          \\
		\hline
	\end{tabular}
	\caption{Testing Tools}
\end{table}


\subsubsection{Code Formatting}

To ensure that changes are strictly functional we will run consistent code formatters and linters,
as we establish the standards the settings of each may change. A code formatter will align the written code
based on the configured settings, this eliminates redundant changes from showing in any Git Diffs e.g. whitespace additions.

\begin{table}[h!]
	\centering
	\begin{tabular}{|c|c|}
		\hline
		\textcolor{McMasterMaroon}{\textbf{Language}} &
		\textcolor{McMasterMaroon}{\textbf{Formatter}}                  \\
		\hline
		Python                                        & Ruff            \\
		\hline
		TypeScript/JavaScript                         & Prettier/ESLint \\
		\hline
		C/C++                                         & clang-format    \\
		\hline
		LaTeX                                         & latexindent     \\
		\hline
	\end{tabular}
	\caption{Formatting Tools}
\end{table}






\clearpage
\subsection{Group Roles}

Below is a table outlining the rough group responsibilities. These are subject to change over product development.

\begin{table}[h!]
	\centering
	\begin{tabularx}{\linewidth}{|Y|Y|}
		\hline
		\textcolor{McMasterMaroon}{\textbf{Group Members}} &
		\textcolor{McMasterMaroon}{\textbf{Roles}}                       \\
		\hline
		Anish                                              &
		\begin{minipage}[c]{\linewidth} % vertically centered content
			\vspace{2mm}
			\begin{itemize}[leftmargin=*, labelsep=3pt]
				\item Documentation
				\item Mechanical design/implementation
				\item Electrical design/implementation
			\end{itemize}
			\vspace{2mm}
		\end{minipage}  \\
		\hline

		Chris                                              &
		\begin{minipage}[c]{\linewidth} % vertically centered content
			\vspace{2mm}
			\begin{itemize}[leftmargin=*, labelsep=3pt]
				\item Electrical design/implementation
				\item Software design/implementation
			\end{itemize}
			\vspace{2mm}
		\end{minipage}  \\
		\hline

		Jacqueline                                         &
		\begin{minipage}[c]{\linewidth} % vertically centered content
			\vspace{2mm}
			\begin{itemize}[leftmargin=*, labelsep=3pt]
				\item Software design/implementation
				\item Electrical design/implementation
			\end{itemize}
			\vspace{2mm}
		\end{minipage}  \\
		\hline

		Mitchel                                            &
		\begin{minipage}[c]{\linewidth} % vertically centered content
			\vspace{2mm}
			\begin{itemize}[leftmargin=*, labelsep=3pt]
				\item Mechanical design/implementation
				\item Electrical design/implementation
			\end{itemize}
			\vspace{2mm}
		\end{minipage} \\
		\hline

		Shivom                                             &
		\begin{minipage}[c]{\linewidth} % vertically centered content
			\vspace{2mm}
			\begin{itemize}[leftmargin=*, labelsep=3pt]
				\item Software Design/implementation
				\item Firmware and Configurator
				\item Electrical design/implementation
			\end{itemize}
			\vspace{2mm}
		\end{minipage} \\
		\hline
	\end{tabularx}
	\caption{Group roles}
\end{table}

\clearpage
\subsection{Technology}

\begin{table}[h!]
	\centering
	\renewcommand{\arraystretch}{1.2} % extra row spacing
	\centering
	\begin{tabularx}{\linewidth}{|Y|Y|}
		\hline
		\textcolor{McMasterMaroon}{\textbf{Software}} & \textcolor{McMasterMaroon}{\textbf{Uses}}                                              \\
		\hline
		GitHub                                        & For any code and version control.                                                      \\
		\hline
		Google Drive                                  & Any collaborative documents and shared files. Used for CAD/mechanical version control. \\
		\hline
		Autodesk Inventor                             & CAD for the housing of the parent and children modules, and for any custom components. \\
		\hline
		STM32CubeIDE, VSCode, Vim                     & IDE for code and microcontrollers.                                                     \\
		\hline
		C/C++                                         & Language for firmware.                                                                 \\
		\hline
		TypeScript/JavaScript                         & UI language for configurator application.                                              \\
		\hline
		Python                                        & Miscellaneous scripting language.                                                      \\
		\hline
		KiCad, EasyEDA, Altium                        & Software for PCB design.                                                               \\
		\hline
		PrusaSlicer, Bambu, Studio                    & Software for 3D printing.                                                              \\
		\hline
		Docker                                        & Containerizing code to run on multiple machines (MacOS, Linux, Windows).               \\
		\hline
		Heroku/Render                                 & Website hosting platform for configurator application.                                 \\
		\hline
		Figma, Mermaid, MSPaint, Procreate            & Softwares used for systems diagramming.                                                \\
		\hline
		LaTeX                                         & Tool used for documentation.                                                           \\
		\hline
	\end{tabularx}
	\caption{Software and their uses}
\end{table}


\begin{table}[h!]
	\renewcommand{\arraystretch}{1.2} % extra row spacing
	\centering
	\begin{tabularx}{\linewidth}{|Y|Y|}
		\hline
		\textcolor{McMasterMaroon}{\textbf{Hardware}}                                   & \textcolor{McMasterMaroon}{\textbf{Uses}}                                    \\
		\hline
		3D Printers (Bambu Lab A1, Thode Makerspace: Prusa MK4S)                        & To print the controller children module casing and any custom components.    \\
		\hline
		STM32F Series Boards                                                            & Controller for the main central unit.                                        \\
		\hline
		ESP32s Series Boards, Raspberry Pi Pico                                         & Module-level control board (Bluetooth connection or wired to central unit).  \\
		\hline
		Arcade Buttons/Joysticks                                                        & Hardware for inputs from children modules.                                   \\
		\hline
		Miscellaneous Electrical Components (Capacitors, Resistors, Power Supply, etc.) & Any electrical components required to build the controller children modules. \\
		\hline
		Breadboards                                                                     & Used in concept design and design verification.                              \\
		\hline
		Breakout Boards                                                                 & Used for making custom cables for USB, micro-USB .                           \\
		\hline
		Bluetooth Card (Wi-Fi 5 enabled)                                                & For communication between children and parent boards.                        \\
		\hline
		Oscilloscope/Multimeter                                                         & Troubleshooting and testing.                                                 \\
		\hline
		Cables (USB, Power cable)                                                       & Used for power and data transfer.                                            \\
		\hline
	\end{tabularx}
	\caption{Hardware and their uses}
\end{table}

\end{document}
